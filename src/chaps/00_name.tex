\section{Name and Acronym}
\label{sec:NameAndAcronym}
\subsection{Outcome}
\begin{description}
  \item[KCMQ:] Knack Central Macquarie University. The mission statement and
  many engineering themes revolve around \emph{knack}, and so we felt reflecting
  this in the name would be a good thing to tie in with the overall branding. It
  is also partly humourous, the Dilbert episode ``The Knack''.
\end{description}

Logo needs to be a simple vector of some description involving at most 4
colours, but multiple tones of colour are permissible. This is to keep printing
costs down, and to make the logo recogniseable.

Logo concepts include circuitry and tools to form letters for the acronym
``KCMQ'' with the motto being: ``Engineering Motto: Challenge Accepted''.

% \begin{figure}[!hbt]
% \label{fig:LogoIdea}
% \begin{tikzpicture}
%   \node[font=\big] at (1,1) {KCMQ};
% \end{tikzpicture}
% \caption{Logo Concept Idea}
% \end{figure}

\subsection{Other candidates}
\begin{description}
  \item[MESS:] Macquarie Engineering Student Society
  \item[MESS:] Macquarie Engineering and Science Society
  \item[MES:] Macquarie Engineering Society
  \item[SAME:] Students At Macquarie in Engineering
  \item[EMU:] Engineers/Engineering at Macquarie University
  \item[E=MQ$^2$:] Engineers/Engineering at Macquarie University
  \item[TEAM:] The Engineers At Macquarie University
  \item[MATE:] Macquarie And The Engineers
  \item[MEAL:]
  \item[LAME:]
  \item[MEAT:]
  \item[MEME:]
  \item[MEAD:]
\end{description}

\subsection{Action Items}
\begin{itemize}
  \item Seek objectors to name.
\end{itemize}